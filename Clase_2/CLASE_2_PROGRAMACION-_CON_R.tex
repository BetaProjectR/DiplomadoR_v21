% Options for packages loaded elsewhere
\PassOptionsToPackage{unicode}{hyperref}
\PassOptionsToPackage{hyphens}{url}
%
\documentclass[
  ignorenonframetext,
]{beamer}
\usepackage{pgfpages}
\setbeamertemplate{caption}[numbered]
\setbeamertemplate{caption label separator}{: }
\setbeamercolor{caption name}{fg=normal text.fg}
\beamertemplatenavigationsymbolsempty
% Prevent slide breaks in the middle of a paragraph
\widowpenalties 1 10000
\raggedbottom
\setbeamertemplate{part page}{
  \centering
  \begin{beamercolorbox}[sep=16pt,center]{part title}
    \usebeamerfont{part title}\insertpart\par
  \end{beamercolorbox}
}
\setbeamertemplate{section page}{
  \centering
  \begin{beamercolorbox}[sep=12pt,center]{part title}
    \usebeamerfont{section title}\insertsection\par
  \end{beamercolorbox}
}
\setbeamertemplate{subsection page}{
  \centering
  \begin{beamercolorbox}[sep=8pt,center]{part title}
    \usebeamerfont{subsection title}\insertsubsection\par
  \end{beamercolorbox}
}
\AtBeginPart{
  \frame{\partpage}
}
\AtBeginSection{
  \ifbibliography
  \else
    \frame{\sectionpage}
  \fi
}
\AtBeginSubsection{
  \frame{\subsectionpage}
}
\usepackage{lmodern}
\usepackage{amssymb,amsmath}
\usepackage{ifxetex,ifluatex}
\ifnum 0\ifxetex 1\fi\ifluatex 1\fi=0 % if pdftex
  \usepackage[T1]{fontenc}
  \usepackage[utf8]{inputenc}
  \usepackage{textcomp} % provide euro and other symbols
\else % if luatex or xetex
  \usepackage{unicode-math}
  \defaultfontfeatures{Scale=MatchLowercase}
  \defaultfontfeatures[\rmfamily]{Ligatures=TeX,Scale=1}
\fi
% Use upquote if available, for straight quotes in verbatim environments
\IfFileExists{upquote.sty}{\usepackage{upquote}}{}
\IfFileExists{microtype.sty}{% use microtype if available
  \usepackage[]{microtype}
  \UseMicrotypeSet[protrusion]{basicmath} % disable protrusion for tt fonts
}{}
\makeatletter
\@ifundefined{KOMAClassName}{% if non-KOMA class
  \IfFileExists{parskip.sty}{%
    \usepackage{parskip}
  }{% else
    \setlength{\parindent}{0pt}
    \setlength{\parskip}{6pt plus 2pt minus 1pt}}
}{% if KOMA class
  \KOMAoptions{parskip=half}}
\makeatother
\usepackage{xcolor}
\IfFileExists{xurl.sty}{\usepackage{xurl}}{} % add URL line breaks if available
\IfFileExists{bookmark.sty}{\usepackage{bookmark}}{\usepackage{hyperref}}
\hypersetup{
  pdftitle={Diplomado en Análisis de datos con R para la acuicultura},
  hidelinks,
  pdfcreator={LaTeX via pandoc}}
\urlstyle{same} % disable monospaced font for URLs
\newif\ifbibliography
\usepackage{graphicx,grffile}
\makeatletter
\def\maxwidth{\ifdim\Gin@nat@width>\linewidth\linewidth\else\Gin@nat@width\fi}
\def\maxheight{\ifdim\Gin@nat@height>\textheight\textheight\else\Gin@nat@height\fi}
\makeatother
% Scale images if necessary, so that they will not overflow the page
% margins by default, and it is still possible to overwrite the defaults
% using explicit options in \includegraphics[width, height, ...]{}
\setkeys{Gin}{width=\maxwidth,height=\maxheight,keepaspectratio}
% Set default figure placement to htbp
\makeatletter
\def\fps@figure{htbp}
\makeatother
\setlength{\emergencystretch}{3em} % prevent overfull lines
\providecommand{\tightlist}{%
  \setlength{\itemsep}{0pt}\setlength{\parskip}{0pt}}
\setcounter{secnumdepth}{-\maxdimen} % remove section numbering

\title{Diplomado en Análisis de datos con R para la acuicultura}
\subtitle{CLASE 2 PROGRAMACION CON R}
\author{true}
\date{20/4/2021}

\begin{document}
\frame{\titlepage}

\begin{frame}{}
\protect\hypertarget{section}{}

\textbf{PLAN DE CLASE}

\textbf{1).} \textbf{Introducción}

\begin{itemize}
\item
  \textbf{Preguntas al curso.}
\item
  \textbf{Estudio de caso.}
\item
  \textbf{Investigación reproducible.}
\item
  \textbf{Software para el análisis de datos.}
\end{itemize}

\textbf{2).} \textbf{Práctica con R y Rstudio cloud}

\begin{itemize}
\item
  \textbf{Iniciar un proyecto de análisis de datos con R.}
\item
  \textbf{Escribir un código de programación o script.}
\item
  \textbf{Familiarizarse con manipulación de objetos y datos.}
\end{itemize}

\end{frame}

\begin{frame}{}
\protect\hypertarget{section-1}{}

\textbf{Introducción}

\textbf{\emph{Clase 2 -- Programación con R}}

\end{frame}

\begin{frame}{}
\protect\hypertarget{section-2}{}

\textbf{PREGUNTAS AL CURSO}

\textbf{1).} \textbf{¿ Por qué deseas hacer análisis de datos con R?}

\textbf{2).} \textbf{¿ Qué problemas has tenido cuando quieres rehacer
un análisis de datos a partir de un set de datos antiguo o en el que no
has trabajado por mucho tiempo?}

\end{frame}

\begin{frame}{\textbf{ESTUDIO DE CASO: Bloom de algas marzo 2016}}
\protect\hypertarget{estudio-de-caso-bloom-de-algas-marzo-2016}{}

\begin{figure}
\centering
\includegraphics{C:/Users/angel/Google Drive/Chile_2021/Diplomados_2021/Diplomado Primer cuatrimestre_2021/CLASES_GITHUB/Imagenes_PPT/CLASE2/ALGAS1.png}
\caption{Blomm de algas}
\end{figure}

\textbf{CONSECUENCIAS ACUICULTURA}

\textbf{Acuicultura} 45 centros de cultivo 14 empresas

\textbf{Mortalidad} +25 millones de peces 40 mil toneladas

\textbf{Vertimiento de mortalidad al mar} 5.000 toneladas

\begin{figure}
\centering
\includegraphics{C:/Users/angel/Google Drive/Chile_2021/Diplomados_2021/Diplomado Primer cuatrimestre_2021/CLASES_GITHUB/Imagenes_PPT/CLASE1/PORTADA.png}
\caption{portada}
\end{figure}

\begin{figure}
\centering
\includegraphics{C:/Users/angel/Google Drive/Chile_2021/Diplomados_2021/Diplomado Primer cuatrimestre_2021/CLASES_GITHUB/LOGOS_PUCV/Escudo_PUCV.jpg}
\caption{Logo\_PUCV}
\end{figure}

\end{frame}

\end{document}
