% Options for packages loaded elsewhere
\PassOptionsToPackage{unicode}{hyperref}
\PassOptionsToPackage{hyphens}{url}
%
\documentclass[
  ignorenonframetext,
]{beamer}
\usepackage{pgfpages}
\setbeamertemplate{caption}[numbered]
\setbeamertemplate{caption label separator}{: }
\setbeamercolor{caption name}{fg=normal text.fg}
\beamertemplatenavigationsymbolsempty
% Prevent slide breaks in the middle of a paragraph
\widowpenalties 1 10000
\raggedbottom
\setbeamertemplate{part page}{
  \centering
  \begin{beamercolorbox}[sep=16pt,center]{part title}
    \usebeamerfont{part title}\insertpart\par
  \end{beamercolorbox}
}
\setbeamertemplate{section page}{
  \centering
  \begin{beamercolorbox}[sep=12pt,center]{part title}
    \usebeamerfont{section title}\insertsection\par
  \end{beamercolorbox}
}
\setbeamertemplate{subsection page}{
  \centering
  \begin{beamercolorbox}[sep=8pt,center]{part title}
    \usebeamerfont{subsection title}\insertsubsection\par
  \end{beamercolorbox}
}
\AtBeginPart{
  \frame{\partpage}
}
\AtBeginSection{
  \ifbibliography
  \else
    \frame{\sectionpage}
  \fi
}
\AtBeginSubsection{
  \frame{\subsectionpage}
}
\usepackage{lmodern}
\usepackage{amssymb,amsmath}
\usepackage{ifxetex,ifluatex}
\ifnum 0\ifxetex 1\fi\ifluatex 1\fi=0 % if pdftex
  \usepackage[T1]{fontenc}
  \usepackage[utf8]{inputenc}
  \usepackage{textcomp} % provide euro and other symbols
\else % if luatex or xetex
  \usepackage{unicode-math}
  \defaultfontfeatures{Scale=MatchLowercase}
  \defaultfontfeatures[\rmfamily]{Ligatures=TeX,Scale=1}
\fi
% Use upquote if available, for straight quotes in verbatim environments
\IfFileExists{upquote.sty}{\usepackage{upquote}}{}
\IfFileExists{microtype.sty}{% use microtype if available
  \usepackage[]{microtype}
  \UseMicrotypeSet[protrusion]{basicmath} % disable protrusion for tt fonts
}{}
\makeatletter
\@ifundefined{KOMAClassName}{% if non-KOMA class
  \IfFileExists{parskip.sty}{%
    \usepackage{parskip}
  }{% else
    \setlength{\parindent}{0pt}
    \setlength{\parskip}{6pt plus 2pt minus 1pt}}
}{% if KOMA class
  \KOMAoptions{parskip=half}}
\makeatother
\usepackage{xcolor}
\IfFileExists{xurl.sty}{\usepackage{xurl}}{} % add URL line breaks if available
\IfFileExists{bookmark.sty}{\usepackage{bookmark}}{\usepackage{hyperref}}
\hypersetup{
  pdftitle={Clase 7 Manipulación de bases de datos con Tidiverse},
  hidelinks,
  pdfcreator={LaTeX via pandoc}}
\urlstyle{same} % disable monospaced font for URLs
\newif\ifbibliography
\usepackage{color}
\usepackage{fancyvrb}
\newcommand{\VerbBar}{|}
\newcommand{\VERB}{\Verb[commandchars=\\\{\}]}
\DefineVerbatimEnvironment{Highlighting}{Verbatim}{commandchars=\\\{\}}
% Add ',fontsize=\small' for more characters per line
\usepackage{framed}
\definecolor{shadecolor}{RGB}{248,248,248}
\newenvironment{Shaded}{\begin{snugshade}}{\end{snugshade}}
\newcommand{\AlertTok}[1]{\textcolor[rgb]{0.94,0.16,0.16}{#1}}
\newcommand{\AnnotationTok}[1]{\textcolor[rgb]{0.56,0.35,0.01}{\textbf{\textit{#1}}}}
\newcommand{\AttributeTok}[1]{\textcolor[rgb]{0.77,0.63,0.00}{#1}}
\newcommand{\BaseNTok}[1]{\textcolor[rgb]{0.00,0.00,0.81}{#1}}
\newcommand{\BuiltInTok}[1]{#1}
\newcommand{\CharTok}[1]{\textcolor[rgb]{0.31,0.60,0.02}{#1}}
\newcommand{\CommentTok}[1]{\textcolor[rgb]{0.56,0.35,0.01}{\textit{#1}}}
\newcommand{\CommentVarTok}[1]{\textcolor[rgb]{0.56,0.35,0.01}{\textbf{\textit{#1}}}}
\newcommand{\ConstantTok}[1]{\textcolor[rgb]{0.00,0.00,0.00}{#1}}
\newcommand{\ControlFlowTok}[1]{\textcolor[rgb]{0.13,0.29,0.53}{\textbf{#1}}}
\newcommand{\DataTypeTok}[1]{\textcolor[rgb]{0.13,0.29,0.53}{#1}}
\newcommand{\DecValTok}[1]{\textcolor[rgb]{0.00,0.00,0.81}{#1}}
\newcommand{\DocumentationTok}[1]{\textcolor[rgb]{0.56,0.35,0.01}{\textbf{\textit{#1}}}}
\newcommand{\ErrorTok}[1]{\textcolor[rgb]{0.64,0.00,0.00}{\textbf{#1}}}
\newcommand{\ExtensionTok}[1]{#1}
\newcommand{\FloatTok}[1]{\textcolor[rgb]{0.00,0.00,0.81}{#1}}
\newcommand{\FunctionTok}[1]{\textcolor[rgb]{0.00,0.00,0.00}{#1}}
\newcommand{\ImportTok}[1]{#1}
\newcommand{\InformationTok}[1]{\textcolor[rgb]{0.56,0.35,0.01}{\textbf{\textit{#1}}}}
\newcommand{\KeywordTok}[1]{\textcolor[rgb]{0.13,0.29,0.53}{\textbf{#1}}}
\newcommand{\NormalTok}[1]{#1}
\newcommand{\OperatorTok}[1]{\textcolor[rgb]{0.81,0.36,0.00}{\textbf{#1}}}
\newcommand{\OtherTok}[1]{\textcolor[rgb]{0.56,0.35,0.01}{#1}}
\newcommand{\PreprocessorTok}[1]{\textcolor[rgb]{0.56,0.35,0.01}{\textit{#1}}}
\newcommand{\RegionMarkerTok}[1]{#1}
\newcommand{\SpecialCharTok}[1]{\textcolor[rgb]{0.00,0.00,0.00}{#1}}
\newcommand{\SpecialStringTok}[1]{\textcolor[rgb]{0.31,0.60,0.02}{#1}}
\newcommand{\StringTok}[1]{\textcolor[rgb]{0.31,0.60,0.02}{#1}}
\newcommand{\VariableTok}[1]{\textcolor[rgb]{0.00,0.00,0.00}{#1}}
\newcommand{\VerbatimStringTok}[1]{\textcolor[rgb]{0.31,0.60,0.02}{#1}}
\newcommand{\WarningTok}[1]{\textcolor[rgb]{0.56,0.35,0.01}{\textbf{\textit{#1}}}}
\setlength{\emergencystretch}{3em} % prevent overfull lines
\providecommand{\tightlist}{%
  \setlength{\itemsep}{0pt}\setlength{\parskip}{0pt}}
\setcounter{secnumdepth}{-\maxdimen} % remove section numbering

\title{Clase 7 Manipulación de bases de datos con Tidiverse}
\subtitle{Diplomado en Análisis de datos con R para la acuicultura}
\author{true}
\date{04 mayo 2021}

\begin{document}
\frame{\titlepage}

\begin{frame}{}
\protect\hypertarget{section}{}

\textbf{PLAN DE CLASE}

\textbf{1).} \textbf{Introducción.}

\begin{itemize}
\item
  \textbf{Preguntas al curso.}
\item
  \textbf{Estudio de caso.}
\end{itemize}

\textbf{2).} \textbf{Práctica con R y Rstudio cloud.}

\begin{itemize}
\item
  \textbf{Manipula y genera nuevas bases de datos y gráficos a partir de
  comandos de Tidyverse.}
\item
  \textbf{Escribir un código de programación} o \textbf{\emph{script}.}
\item
  \textbf{Elaborar un reporte html con Rmarkdown.}
\end{itemize}

\end{frame}

\begin{frame}{}
\protect\hypertarget{section-1}{}

\textbf{Introducción}

\textbf{\emph{Clase 7 -- Manipulación de bases de datos}}

\end{frame}

\begin{frame}{}
\protect\hypertarget{section-2}{}

\textbf{PREGUNTAS AL CURSO}

\textbf{1).} \textbf{Señale 3 razones de la importancia de filtrar
observaciones por algún criterio, hacer subconjuntos de datos y generar
variables derivadas con información de variables existentes en el set de
datos}.

\textbf{2).} \textbf{Desde su experiencia, comente sobre que tipo de
herramientas ha usado en sus análisis de datos, para filtrar o generar
subconjuntos de bases de datos}.

\end{frame}

\begin{frame}{}
\protect\hypertarget{section-3}{}

\textbf{ESTUDIO DE CASO}

\textbf{¿Hay datos faltantes en el set de datos, si hay, cuántos son?}

\textbf{¿ Se podría en R sumar datos NA con números, come sería el
resultado de adicionar estos tipos de datos?}

\begin{block}{}

\textbf{MANIPULACIÓN DE BASES DE DATOS}

Es una de las etapas que más tiempo demanda antes de realizar el proceso
de analizar los datos; ya que implica dar el formato adecuado a nuestro
set de datos. Algunos aspectos a considerar a la hora de manipular bases
de datos son:

\begin{itemize}
\item
  Aplicar filtros
\item
  Remover o imputar datos faltantes
\item
  Usar agrupamientos por algún criterio
\item
  Seleccionar variables
\item
  Generar variables derivadas de la ya existentes
\end{itemize}

\textbf{\emph{Una herramienta útil para la manipulación de datos es
Tidyverse.}}

\end{block}

\begin{block}{}

\textbf{¿ CÓMO FUNCIONA LA MANIPULACIÓN DE BASES DE DATOS CON
TIDIVERSE?}

\begin{figure}

{\centering \includegraphics[width=0.9\linewidth]{Tidyverse} 

}

\caption{Figura tomada de DataCamp}\label{fig:unnamed-chunk-1}
\end{figure}

\end{block}

\end{frame}

\begin{frame}{}
\protect\hypertarget{section-6}{}

\textbf{¿CÓMO MANIPULAR LOS DATA FRAMES?}

El paquete \textbf{dplyr} del tidyverse posee funciones que realizan
algunas de las operaciones más comunes cuando se trabaja con los
\textbf{\emph{data frames}}, permitiendo manipular de una forma ágil la
base de datos que queremos analizar \emph{a posteriori}.

Veamos algunas \textbf{funciones claves}:

\begin{itemize}
\item
  Para cambiar la tabla de datos agregando una nueva columna, utilizamos
  \textbf{mutate()}.
\item
  Para filtrar la tabla de datos a un subconjunto de filas, utilizamos
  \textbf{filter()}.
\item
  Para subdividir los datos seleccionando columnas específicas, usamos
  \textbf{select()}.
\end{itemize}

\end{frame}

\begin{frame}{}
\protect\hypertarget{section-7}{}

\textbf{PRODUCCIÓN Y DISPONIBILIDAD NACIONAL DE OVAS 2016-2017}

El año 2017, la producción nacional de ovas aumentó en un 9\% en
relación al año 2016. El principal proveedor de ovas de salmónidos es
Islandia.

\begin{figure}

{\centering \includegraphics[width=0.8\linewidth]{Ovas} 

}

\caption{Figura tomada de informe de Sernapesca}\label{fig:unnamed-chunk-2}
\end{figure}

\end{frame}

\begin{frame}[fragile]{}
\protect\hypertarget{section-8}{}

\textbf{FILTRAR DATOS DE PRODUCCIÓN Y DISPONIBILIDAD NACIONAL DE OVAS
DEL AÑO 2016}

\begin{Shaded}
\begin{Highlighting}[]
\KeywordTok{filter}\NormalTok{(Year }\OperatorTok{==}\StringTok{ }\DecValTok{2016}\NormalTok{)}
\end{Highlighting}
\end{Shaded}

\begin{table}

\caption{\label{tab:table}Producción y disponibilidad de Nacional de ovas 2016-2017 (en millones de ovas).}
\centering
\fontsize{20}{22}\selectfont
\begin{tabular}[t]{l|r|r|r|r}
\hline
Especies & Year & ProdNac & DispOvas & Importacion\\
\hline
\cellcolor{blue}{\textcolor{white}{\textbf{Salmón del Atlántico}}} & \cellcolor{blue}{\textcolor{white}{\textbf{2016}}} & \cellcolor{blue}{\textcolor{white}{\textbf{361.24}}} & \cellcolor{blue}{\textcolor{white}{\textbf{478.47}}} & \cellcolor{blue}{\textcolor{white}{\textbf{3.32}}}\\
\hline
\cellcolor{blue}{\textcolor{white}{\textbf{Salmón Plateado}}} & \cellcolor{blue}{\textcolor{white}{\textbf{2016}}} & \cellcolor{blue}{\textcolor{white}{\textbf{124.60}}} & \cellcolor{blue}{\textcolor{white}{\textbf{186.83}}} & \cellcolor{blue}{\textcolor{white}{\textbf{0.00}}}\\
\hline
\cellcolor{blue}{\textcolor{white}{\textbf{Trucha Arcoiris}}} & \cellcolor{blue}{\textcolor{white}{\textbf{2016}}} & \cellcolor{blue}{\textcolor{white}{\textbf{101.04}}} & \cellcolor{blue}{\textcolor{white}{\textbf{187.40}}} & \cellcolor{blue}{\textcolor{white}{\textbf{0.00}}}\\
\hline
Salmón del Atlántico & 2017 & 426.82 & 587.17 & 7.54\\
\hline
Salmón Plateado & 2017 & 109.15 & 206.68 & 0.00\\
\hline
Trucha Arcoiris & 2017 & 106.25 & 164.14 & 0.00\\
\hline
\end{tabular}
\end{table}

\end{frame}

\begin{frame}[fragile]{}
\protect\hypertarget{section-9}{}

\textbf{AGRUPAR DATOS DE PRODUCCIÓN Y DISPONIBILIDAD NACIONAL DE OVAS
PARA LOS AÑOS EVALUADOS}

\begin{Shaded}
\begin{Highlighting}[]
\KeywordTok{group_by}\NormalTok{(Year)}
\end{Highlighting}
\end{Shaded}

\begin{table}

\caption{\label{tab:unnamed-chunk-5}Producción y disponibilidad de Nacional de ovas 2016-2017 (en millones de ovas).}
\centering
\fontsize{20}{22}\selectfont
\begin{tabular}[t]{l|r|r|r|r}
\hline
Especies & Year & ProdNac & DispOvas & Importacion\\
\hline
\cellcolor{blue}{\textcolor{white}{\textbf{Salmón del Atlántico}}} & \cellcolor{blue}{\textcolor{white}{\textbf{2016}}} & \cellcolor{blue}{\textcolor{white}{\textbf{361.24}}} & \cellcolor{blue}{\textcolor{white}{\textbf{478.47}}} & \cellcolor{blue}{\textcolor{white}{\textbf{3.32}}}\\
\hline
\cellcolor{blue}{\textcolor{white}{\textbf{Salmón Plateado}}} & \cellcolor{blue}{\textcolor{white}{\textbf{2016}}} & \cellcolor{blue}{\textcolor{white}{\textbf{124.60}}} & \cellcolor{blue}{\textcolor{white}{\textbf{186.83}}} & \cellcolor{blue}{\textcolor{white}{\textbf{0.00}}}\\
\hline
\cellcolor{blue}{\textcolor{white}{\textbf{Trucha Arcoiris}}} & \cellcolor{blue}{\textcolor{white}{\textbf{2016}}} & \cellcolor{blue}{\textcolor{white}{\textbf{101.04}}} & \cellcolor{blue}{\textcolor{white}{\textbf{187.40}}} & \cellcolor{blue}{\textcolor{white}{\textbf{0.00}}}\\
\hline
\cellcolor{blue}{\textcolor{white}{\textbf{Salmón del Atlántico}}} & \cellcolor{blue}{\textcolor{white}{\textbf{2017}}} & \cellcolor{blue}{\textcolor{white}{\textbf{426.82}}} & \cellcolor{blue}{\textcolor{white}{\textbf{587.17}}} & \cellcolor{blue}{\textcolor{white}{\textbf{7.54}}}\\
\hline
\cellcolor{blue}{\textcolor{white}{\textbf{Salmón Plateado}}} & \cellcolor{blue}{\textcolor{white}{\textbf{2017}}} & \cellcolor{blue}{\textcolor{white}{\textbf{109.15}}} & \cellcolor{blue}{\textcolor{white}{\textbf{206.68}}} & \cellcolor{blue}{\textcolor{white}{\textbf{0.00}}}\\
\hline
\cellcolor{blue}{\textcolor{white}{\textbf{Trucha Arcoiris}}} & \cellcolor{blue}{\textcolor{white}{\textbf{2017}}} & \cellcolor{blue}{\textcolor{white}{\textbf{106.25}}} & \cellcolor{blue}{\textcolor{white}{\textbf{164.14}}} & \cellcolor{blue}{\textcolor{white}{\textbf{0.00}}}\\
\hline
\end{tabular}
\end{table}

\end{frame}

\begin{frame}[fragile]{}
\protect\hypertarget{section-10}{}

\textbf{SELECCIONAR VARIABLES DE LOS DATOS DE PRODUCCIÓN Y
DISPONIBILIDAD NACIONAL DE OVAS}

\begin{Shaded}
\begin{Highlighting}[]
\KeywordTok{select}\NormalTok{(Especies, Year,ProdNac)}
\end{Highlighting}
\end{Shaded}

\begin{table}

\caption{\label{tab:unnamed-chunk-7}Producción y disponibilidad de Nacional de ovas 2016-2017 (en millones de ovas).}
\centering
\fontsize{20}{22}\selectfont
\begin{tabular}[t]{>{}l|>{}r|>{}r|r|r}
\hline
Especies & Year & ProdNac & DispOvas & Importacion\\
\hline
\cellcolor{blue}{\textcolor{white}{\textbf{Salmón del Atlántico}}} & \cellcolor{blue}{\textcolor{white}{\textbf{2016}}} & \cellcolor{blue}{\textcolor{white}{\textbf{361.24}}} & 478.47 & 3.32\\
\hline
\cellcolor{blue}{\textcolor{white}{\textbf{Salmón Plateado}}} & \cellcolor{blue}{\textcolor{white}{\textbf{2016}}} & \cellcolor{blue}{\textcolor{white}{\textbf{124.60}}} & 186.83 & 0.00\\
\hline
\cellcolor{blue}{\textcolor{white}{\textbf{Trucha Arcoiris}}} & \cellcolor{blue}{\textcolor{white}{\textbf{2016}}} & \cellcolor{blue}{\textcolor{white}{\textbf{101.04}}} & 187.40 & 0.00\\
\hline
\cellcolor{blue}{\textcolor{white}{\textbf{Salmón del Atlántico}}} & \cellcolor{blue}{\textcolor{white}{\textbf{2017}}} & \cellcolor{blue}{\textcolor{white}{\textbf{426.82}}} & 587.17 & 7.54\\
\hline
\cellcolor{blue}{\textcolor{white}{\textbf{Salmón Plateado}}} & \cellcolor{blue}{\textcolor{white}{\textbf{2017}}} & \cellcolor{blue}{\textcolor{white}{\textbf{109.15}}} & 206.68 & 0.00\\
\hline
\cellcolor{blue}{\textcolor{white}{\textbf{Trucha Arcoiris}}} & \cellcolor{blue}{\textcolor{white}{\textbf{2017}}} & \cellcolor{blue}{\textcolor{white}{\textbf{106.25}}} & 164.14 & 0.00\\
\hline
\end{tabular}
\end{table}

\end{frame}

\begin{frame}[fragile]{}
\protect\hypertarget{section-11}{}

\textbf{MEDIDAS RESUMEN POR VARIABLE}

\begin{Shaded}
\begin{Highlighting}[]
\KeywordTok{summarize}\NormalTok{(}\DataTypeTok{Min_ProdNac =} \KeywordTok{min}\NormalTok{(ProdNac), }\DataTypeTok{Max_ProdNac =} \KeywordTok{max}\NormalTok{(ProdNac))}
\end{Highlighting}
\end{Shaded}

\begin{verbatim}
## # A tibble: 1 x 2
##   Min_ProdNac Max_ProdNac
##         <dbl>       <dbl>
## 1        101.        427.
\end{verbatim}

\begin{table}

\caption{\label{tab:unnamed-chunk-10}Producción y disponibilidad de Nacional de ovas 2016-2017 (en millones de ovas).}
\centering
\fontsize{20}{22}\selectfont
\begin{tabular}[t]{l|r|>{}r|r|r}
\hline
Especies & Year & ProdNac & DispOvas & Importacion\\
\hline
Salmón del Atlántico & 2016 & \cellcolor{blue}{\textcolor{white}{\textbf{361.24}}} & 478.47 & 3.32\\
\hline
Salmón Plateado & 2016 & \cellcolor{blue}{\textcolor{white}{\textbf{124.60}}} & 186.83 & 0.00\\
\hline
Trucha Arcoiris & 2016 & \cellcolor{blue}{\textcolor{white}{\textbf{101.04}}} & 187.40 & 0.00\\
\hline
Salmón del Atlántico & 2017 & \cellcolor{blue}{\textcolor{white}{\textbf{426.82}}} & 587.17 & 7.54\\
\hline
Salmón Plateado & 2017 & \cellcolor{blue}{\textcolor{white}{\textbf{109.15}}} & 206.68 & 0.00\\
\hline
Trucha Arcoiris & 2017 & \cellcolor{blue}{\textcolor{white}{\textbf{106.25}}} & 164.14 & 0.00\\
\hline
\end{tabular}
\end{table}

\end{frame}

\begin{frame}{}
\protect\hypertarget{section-12}{}

\textbf{REPRESENTACIÓN GRAFICA USANDO GGPLOT2}

\begin{center}\includegraphics{Clase7_files/figure-beamer/unnamed-chunk-11-1} \end{center}

\end{frame}

\begin{frame}{}
\protect\hypertarget{section-13}{}

\textbf{Práctica}

\textbf{\emph{Clase 7 -- Manipulación de bases de datos}}

\end{frame}

\begin{frame}{}
\protect\hypertarget{section-14}{}

\textbf{TRABAJO EN SALAS}

\textbf{1).} \textbf{Guía de trabajo programación con R disponible en
drive.}

\textbf{2).} \textbf{La tarea se realiza en Rstudio.cloud}.

Ingresa al siguiente proyecto de
\textbf{\href{https://rstudio.cloud/spaces/135178/project/2447826/}{Rstudio.Cloud}}

\end{frame}

\begin{frame}{}
\protect\hypertarget{section-15}{}

\textbf{RESUMEN DE LA CLASE}

\begin{itemize}
\item
  Revisar ventajas de manipular bases de datos con
  \textbf{\emph{Tidyverse}}.
\item
  Escribir un código de programación con \textbf{\emph{Rmardown}}
\item
  Elaborar diferentes reportes dinámicos.
\end{itemize}

\end{frame}

\end{document}
