% Options for packages loaded elsewhere
\PassOptionsToPackage{unicode}{hyperref}
\PassOptionsToPackage{hyphens}{url}
%
\documentclass[
]{article}
\usepackage{lmodern}
\usepackage{amssymb,amsmath}
\usepackage{ifxetex,ifluatex}
\ifnum 0\ifxetex 1\fi\ifluatex 1\fi=0 % if pdftex
  \usepackage[T1]{fontenc}
  \usepackage[utf8]{inputenc}
  \usepackage{textcomp} % provide euro and other symbols
\else % if luatex or xetex
  \usepackage{unicode-math}
  \defaultfontfeatures{Scale=MatchLowercase}
  \defaultfontfeatures[\rmfamily]{Ligatures=TeX,Scale=1}
\fi
% Use upquote if available, for straight quotes in verbatim environments
\IfFileExists{upquote.sty}{\usepackage{upquote}}{}
\IfFileExists{microtype.sty}{% use microtype if available
  \usepackage[]{microtype}
  \UseMicrotypeSet[protrusion]{basicmath} % disable protrusion for tt fonts
}{}
\makeatletter
\@ifundefined{KOMAClassName}{% if non-KOMA class
  \IfFileExists{parskip.sty}{%
    \usepackage{parskip}
  }{% else
    \setlength{\parindent}{0pt}
    \setlength{\parskip}{6pt plus 2pt minus 1pt}}
}{% if KOMA class
  \KOMAoptions{parskip=half}}
\makeatother
\usepackage{xcolor}
\IfFileExists{xurl.sty}{\usepackage{xurl}}{} % add URL line breaks if available
\IfFileExists{bookmark.sty}{\usepackage{bookmark}}{\usepackage{hyperref}}
\hypersetup{
  pdftitle={Diplomado en Análisis de datos con R para la acuicultura},
  hidelinks,
  pdfcreator={LaTeX via pandoc}}
\urlstyle{same} % disable monospaced font for URLs
\usepackage[margin=1in]{geometry}
\usepackage{graphicx,grffile}
\makeatletter
\def\maxwidth{\ifdim\Gin@nat@width>\linewidth\linewidth\else\Gin@nat@width\fi}
\def\maxheight{\ifdim\Gin@nat@height>\textheight\textheight\else\Gin@nat@height\fi}
\makeatother
% Scale images if necessary, so that they will not overflow the page
% margins by default, and it is still possible to overwrite the defaults
% using explicit options in \includegraphics[width, height, ...]{}
\setkeys{Gin}{width=\maxwidth,height=\maxheight,keepaspectratio}
% Set default figure placement to htbp
\makeatletter
\def\fps@figure{htbp}
\makeatother
\setlength{\emergencystretch}{3em} % prevent overfull lines
\providecommand{\tightlist}{%
  \setlength{\itemsep}{0pt}\setlength{\parskip}{0pt}}
\setcounter{secnumdepth}{-\maxdimen} % remove section numbering

\title{Diplomado en Análisis de datos con R para la acuicultura}
\usepackage{etoolbox}
\makeatletter
\providecommand{\subtitle}[1]{% add subtitle to \maketitle
  \apptocmd{\@title}{\par {\large #1 \par}}{}{}
}
\makeatother
\subtitle{CLASE 1 PRESENTACIÓN}
\author{true}
\date{15/4/2021}

\begin{document}
\maketitle

\hypertarget{section}{%
\subsection{}\label{section}}

\textbf{PLAN DE LA CLASE}

\textbf{1).} \textbf{Palabras de Bienvenida}.

\textbf{2).} \textbf{Presentación de los participantes}.

\textbf{3).} \textbf{Programa del curso}.

\textbf{4).} \textbf{Habilitar recursos de aprendizaje}.

\hypertarget{section-1}{%
\subsection{}\label{section-1}}

\textbf{PROFESORES}

\includegraphics{~/GitHub/DiplomadoR_Acuicultura_v21/Imagenes/Clase1/Fotojose.png}

\textbf{Dr.~José Gallardo}

Profesor adjunto PUCV

Doctor en Ciencias

\includegraphics{~/GitHub/DiplomadoR_Acuicultura_v21/Imagenes/Clase1/Fotoangelica.png}

\textbf{Dr.~María Angélica Rueda}

Investigadora PUCV

Doctor en Ciencias Agropecuarias

\hypertarget{section-2}{%
\subsection{}\label{section-2}}

\textbf{PRESENTACIÓN Y REVISIÓN DEL PROGRAMA DEL CURSO}

\hypertarget{section-3}{%
\subsection{}\label{section-3}}

\textbf{DESCRIPCIÓN Y PRERREQUISITOS }

\hypertarget{section-4}{%
\subsubsection{}\label{section-4}}

El \textbf{\emph{Diplomado en análisis de datos con R para la
acuicultura}} tiene como propósito que los estudiantes desarrollen
habilidades para llevar a cabo un proyecto de análisis de datos e
investigación reproducible en acuicultura usando el lenguaje de
programación \textbf{R}.

\begin{itemize}
\tightlist
\item
  \textbf{Bioestadística.}
\item
  \textbf{Programación en R.}
\end{itemize}

\includegraphics{~/GitHub/DiplomadoR_Acuicultura_v21/Imagenes/Clase1/Prerrequisitos.png}

\hypertarget{section-5}{%
\subsection{}\label{section-5}}

\textbf{FECHAS DE CLASES}

\textbf{Presentación del diplomado y evaluación diagnóstica}

Jueves 15 de abril - 19:00 - 20:00 P.M.

\textbf{Clases sincrónicas}

Martes y jueves de 19:00 - 21:00 P.M.

\textbf{20,22,27,29 de abril}

\textbf{4,6,11,13,18,20,25,27 de mayo}

\textbf{1,3,8,10,15,17,22,24,29 de junio}

\textbf{6,8,13,15,20,22,26 de julio}

\textbf{Cierre del diplomado y entrega de certificados de aprobación}
Jueves 12 de agosto -19:00 - 20:00 P.M.

\hypertarget{section-6}{%
\subsection{}\label{section-6}}

\textbf{CONTENIDOS CENTRALES}

\textbf{UNIDAD 1}

INVESTIGACIÓN REPRODUCIBLE Y ANÁLISIS EXPLORATORIO DE DATOS.

\textbf{UNIDAD 2}

CONTRASTES DE HIPÓTESIS PARAMÉTRICAS Y NO PARAMÉTRICAS.

\textbf{UNIDAD 3}

MODELOS LINEALES Y ANÁLISIS MULTIVARIADO.

\textbf{UNIDAD 4}

PROYECTO PERSONAL Y ANÁLISIS DE DATOS CON \textbf{R}.

\hypertarget{section-7}{%
\subsection{}\label{section-7}}

\textbf{COMPONENTES DE EVALUACIÓN}

\textbf{1).} \textbf{Tareas de autoaprendizaje:} Número variable de
tareas de autoaprendizaje asignadas por el profesor mediante la
plataforma \textbf{\href{https://www.datacamp.com/}{DataCamp}}.

\textbf{2).} \textbf{Tareas de evaluación de competencias:} Cuatro
tareas de resolución de problemas o ejercicios en forma individual, una
para cada unidad.

\textbf{3).} \textbf{Nota de aprobación:} La nota final de aprobación es
un \textbf{4,0} con un 60\% de exigencia, la que se calculará como el
promedio simple de las 4 tareas de evaluación de competencias.

\textbf{4).} \textbf{Calificación por no entrega de tareas:} No entregar
las tareas en el plazo establecido para ello será calificado con la nota
mínima \textbf{(1,0)}.

\textbf{5).} \textbf{Causales de reprobación:} 1) Promedio menor a
\textbf{4,0} en las tareas de evaluación de competencias, b) No realizar
el \textbf{100\%} de las tareas de autoaprendizaje, c) Asistencia a
clases sincrónicas menor al \textbf{75\%}.

\hypertarget{section-8}{%
\subsection{}\label{section-8}}

\textbf{RECURSOS DE APRENDIZAJE}

\hypertarget{section-9}{%
\subsection{}\label{section-9}}

\textbf{CORREO MAIL PUCV}

\includegraphics{~/GitHub/DiplomadoR_Acuicultura_v21/Imagenes/Clase1/correopucv.png}

\hypertarget{section-10}{%
\subsection{}\label{section-10}}

\textbf{FORO DE PREGUNTAS - SLACK}

\includegraphics{~/GitHub/DiplomadoR_Acuicultura_v21/Imagenes/Clase1/Slack.png}

\hypertarget{section-11}{%
\subsection{}\label{section-11}}

\textbf{CUENTA EN DATACAMP}

\includegraphics{~/GitHub/DiplomadoR_Acuicultura_v21/Imagenes/Clase1/Cuentadatacamp.png}

\hypertarget{section-12}{%
\subsection{}\label{section-12}}

\textbf{CUENTA EN R STUDIO CLOUD:} \textbf{\emph{PRÓXIMO MARTES}}

\includegraphics{~/GitHub/DiplomadoR_Acuicultura_v21/Imagenes/Clase1/Rstudiocloud.png}

\hypertarget{section-13}{%
\subsection{}\label{section-13}}

\textbf{TAREA SEMANA 1}

\includegraphics{~/GitHub/DiplomadoR_Acuicultura_v21/Imagenes/Clase1/tareasemana1.png}

\end{document}
