% Options for packages loaded elsewhere
\PassOptionsToPackage{unicode}{hyperref}
\PassOptionsToPackage{hyphens}{url}
%
\documentclass[
  ignorenonframetext,
]{beamer}
\usepackage{pgfpages}
\setbeamertemplate{caption}[numbered]
\setbeamertemplate{caption label separator}{: }
\setbeamercolor{caption name}{fg=normal text.fg}
\beamertemplatenavigationsymbolsempty
% Prevent slide breaks in the middle of a paragraph
\widowpenalties 1 10000
\raggedbottom
\setbeamertemplate{part page}{
  \centering
  \begin{beamercolorbox}[sep=16pt,center]{part title}
    \usebeamerfont{part title}\insertpart\par
  \end{beamercolorbox}
}
\setbeamertemplate{section page}{
  \centering
  \begin{beamercolorbox}[sep=12pt,center]{part title}
    \usebeamerfont{section title}\insertsection\par
  \end{beamercolorbox}
}
\setbeamertemplate{subsection page}{
  \centering
  \begin{beamercolorbox}[sep=8pt,center]{part title}
    \usebeamerfont{subsection title}\insertsubsection\par
  \end{beamercolorbox}
}
\AtBeginPart{
  \frame{\partpage}
}
\AtBeginSection{
  \ifbibliography
  \else
    \frame{\sectionpage}
  \fi
}
\AtBeginSubsection{
  \frame{\subsectionpage}
}
\usepackage{lmodern}
\usepackage{amssymb,amsmath}
\usepackage{ifxetex,ifluatex}
\ifnum 0\ifxetex 1\fi\ifluatex 1\fi=0 % if pdftex
  \usepackage[T1]{fontenc}
  \usepackage[utf8]{inputenc}
  \usepackage{textcomp} % provide euro and other symbols
\else % if luatex or xetex
  \usepackage{unicode-math}
  \defaultfontfeatures{Scale=MatchLowercase}
  \defaultfontfeatures[\rmfamily]{Ligatures=TeX,Scale=1}
\fi
% Use upquote if available, for straight quotes in verbatim environments
\IfFileExists{upquote.sty}{\usepackage{upquote}}{}
\IfFileExists{microtype.sty}{% use microtype if available
  \usepackage[]{microtype}
  \UseMicrotypeSet[protrusion]{basicmath} % disable protrusion for tt fonts
}{}
\makeatletter
\@ifundefined{KOMAClassName}{% if non-KOMA class
  \IfFileExists{parskip.sty}{%
    \usepackage{parskip}
  }{% else
    \setlength{\parindent}{0pt}
    \setlength{\parskip}{6pt plus 2pt minus 1pt}}
}{% if KOMA class
  \KOMAoptions{parskip=half}}
\makeatother
\usepackage{xcolor}
\IfFileExists{xurl.sty}{\usepackage{xurl}}{} % add URL line breaks if available
\IfFileExists{bookmark.sty}{\usepackage{bookmark}}{\usepackage{hyperref}}
\hypersetup{
  pdftitle={Clase 5 Variables aleatorias Discretas},
  hidelinks,
  pdfcreator={LaTeX via pandoc}}
\urlstyle{same} % disable monospaced font for URLs
\newif\ifbibliography
\usepackage{graphicx,grffile}
\makeatletter
\def\maxwidth{\ifdim\Gin@nat@width>\linewidth\linewidth\else\Gin@nat@width\fi}
\def\maxheight{\ifdim\Gin@nat@height>\textheight\textheight\else\Gin@nat@height\fi}
\makeatother
% Scale images if necessary, so that they will not overflow the page
% margins by default, and it is still possible to overwrite the defaults
% using explicit options in \includegraphics[width, height, ...]{}
\setkeys{Gin}{width=\maxwidth,height=\maxheight,keepaspectratio}
% Set default figure placement to htbp
\makeatletter
\def\fps@figure{htbp}
\makeatother
\setlength{\emergencystretch}{3em} % prevent overfull lines
\providecommand{\tightlist}{%
  \setlength{\itemsep}{0pt}\setlength{\parskip}{0pt}}
\setcounter{secnumdepth}{-\maxdimen} % remove section numbering

\title{Clase 5 Variables aleatorias Discretas}
\subtitle{Diplomado en Análisis de datos con R para la acuicultura}
\author{true}
\date{29 abril 2021}

\begin{document}
\frame{\titlepage}

\begin{frame}{}
\protect\hypertarget{section}{}

\textbf{PLAN DE CLASE}

\textbf{1).} \textbf{Introducción}

\begin{itemize}
\item
  \textbf{Preguntas al curso.}
\item
  \textbf{Estudio de caso.}
\end{itemize}

\textbf{2).} \textbf{Práctica con R y Rstudio cloud}

\begin{itemize}
\item
  \textbf{Observa y predice el comportamiento de variables aleatorias
  discretas (Binomial y Bernoulli).}
\item
  \textbf{Escribir un código de programación} o \textbf{\emph{script}}
\item
  \textbf{Elaborar un reporte html con Rmarkdown.}
\end{itemize}

\end{frame}

\begin{frame}{}
\protect\hypertarget{section-1}{}

\textbf{Introducción}

\textbf{\emph{Clase 5 -- Variables discretas}}

\end{frame}

\begin{frame}{}
\protect\hypertarget{section-2}{}

\textbf{PREGUNTAS AL CURSO}

\textbf{1).} \textbf{Señale 3 razones de la importancia de reconocer y
analizar adecuadamente las variables discretas}.

\textbf{2).} \textbf{Señale 3 ejemplos de variables discretas que haya
trabajado o conozca, indique para cada una de ellas que tipo de
distribución tiene la variable}.

\end{frame}

\begin{frame}{}
\protect\hypertarget{section-3}{}

\textbf{ESTUDIO DE CASO}

\textbf{¿Cuál fue la variable en estudio?}

\textbf{¿Qué tipo de distribución usaron?}

\end{frame}

\begin{frame}{}
\protect\hypertarget{section-4}{}

\textbf{VARIABLE ALEATORIA DISCRETA CON DISTRIBUCIÓN BERNOULLI}

VENENO PARALIZANTE DE LOS MARISCOS (VPM)

Producido en Chile por una microalga llamada Alexandrium catenella.

\includegraphics{~/GitHub/DiplomadoR_Acuicultura_v21/Imagenes/Clase5/LetalidadVPM.png}

Intoxicaciones por VPM en Chile 1972-2002 456 enfermos, 30 fallecidos
(total = 486) Letalidad= 30 / 486 = 0.0617

\begin{block}{}

\textbf{DESCRIBIR EL COMPORTAMIENTO DE MORTALIDAD POR VPM}

Ejemplo: Ocurre un evento de marea roja no detectado con A. catenella,
80 personas resultan intoxicadas en todo Chile.

Si la probabilidad de muerte x VPM es de 0,0617.

¿Cuántas personas morirán?

\includegraphics{~/GitHub/DiplomadoR_Acuicultura_v21/Imagenes/Clase5/LetalidadVPM.png}

\begin{block}{}

\textbf{EXPERIMENTO BERNOULLI}

\end{block}

\includegraphics{~/GitHub/DiplomadoR_Acuicultura_v21/Imagenes/Clase5/EnsayoBer.png}

\end{block}

\end{frame}

\begin{frame}{}
\protect\hypertarget{section-7}{}

\textbf{DISTRIBUCIÓN BERNOULLI}

\includegraphics{~/GitHub/DiplomadoR_Acuicultura_v21/Imagenes/Clase5/DistBer.png}

\end{frame}

\begin{frame}{}
\protect\hypertarget{section-8}{}

\textbf{HISTOGRAMA Y FUNCIÓN DE DENSIDAD}

\includegraphics{~/GitHub/DiplomadoR_Acuicultura_v21/Imagenes/Clase5/HistBer.png}

\end{frame}

\begin{frame}{}
\protect\hypertarget{section-9}{}

\textbf{EXPERMIENTO BINOMIAL}

Es un experimento que debe cumplir las siguientes condiciones:

\textbf{1.} El experimento consta de una secuencia de \textbf{n} ensayos
idénticos.

\textbf{2.} En cada ensayo hay dos resultados posibles. A uno de ellos
se le llama \textbf{éxito} y al otro, \textbf{fracaso}.

\textbf{3.} La probabilidad de éxito es constante de un ensayo a otro,
nunca cambia y se denota por \textbf{p}. Por ello, la probabilidad de
fracaso será \textbf{1-p}.

\textbf{4.} Los ensayos son \textbf{independientes}, de modo que el
resultado de cualquiera de ellos \textbf{\emph{no}} influye en el
resultado de cualquier otro ensayo.

\end{frame}

\begin{frame}{}
\protect\hypertarget{section-10}{}

\textbf{VARIABLE ALEATORIA DISCRETA CON DISTRIBUCIÓN BINOMIAL NEGATIVA}

\includegraphics{~/GitHub/DiplomadoR_Acuicultura_v21/Imagenes/Clase5/Binega.png}

\end{frame}

\begin{frame}{}
\protect\hypertarget{section-11}{}

\textbf{SIMULAR VARIABLE ALEATORIA DISCRETA CON DISTRIBUCIÓN BINOMIAL
NEGATIVA}

La abundancia de parásitos como el piojo de mar es una variable distreta
con distribución binomial negativa, esto significa que hay muchos peces
con pocos parásitos\\
(ej= 0 o 1) y poco con muchos parásitos.

\includegraphics{~/GitHub/DiplomadoR_Acuicultura_v21/Imagenes/Clase5/parasitos.png}

\end{frame}

\begin{frame}{}
\protect\hypertarget{section-12}{}

\textbf{HISTOGRAMA Y BOXPLOT}

\includegraphics{~/GitHub/DiplomadoR_Acuicultura_v21/Imagenes/Clase5/caligulus.png}

\end{frame}

\begin{frame}{}
\protect\hypertarget{section-13}{}

\textbf{Práctica con Rmardown}

\textbf{\emph{Clase 5 -- Variables discretas}}

\end{frame}

\begin{frame}{}
\protect\hypertarget{section-14}{}

\textbf{TRABAJO EN SALAS}

\textbf{1).} \textbf{Guía de trabajo programación con R disponible en
drive.}

\textbf{2).} \textbf{La tarea se realiza en Rstudio.cloud}.

Ingresa al siguiente proyecto de
\textbf{\href{https://rstudio.cloud/spaces/135178/project/2447826/}{Rstudio.Cloud}}

\end{frame}

\begin{frame}{}
\protect\hypertarget{section-15}{}

\textbf{RESUMEN DE LA CLASE}

\begin{itemize}
\item
  Revisar ventajas de elaborar reportes dinámicos con
  \textbf{\emph{Rmardown}}.
\item
  Escribir un código de programación con \textbf{\emph{Rmardown}}
\item
  Elaborar diferentes reportes dinámicos.
\end{itemize}

\end{frame}

\end{document}
